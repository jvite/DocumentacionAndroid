\chapter{Marco te'orico}
\section{Herramientas utilizadas}
% Please add the following required packages to your document preamble:
% \usepackage{longtable}
% Note: It may be necessary to compile the document several times to get a multi-page table to line up properly
\begin{longtable}[c]{llll}
\caption{Marco teórico}
\label{Marco-teorico}\\
\hline
\multicolumn{4}{c}{Marco Teórico}                                                                                                                                                                                                                                                                                                                                                                                                                                                                                                                                                                \\ \hline
\endfirsthead
%
\multicolumn{4}{c}%
{{\bfseries Table \thetable\ continued from previous page}} \\
\endhead
%
Lista de Temas                                                                 & Subtemas      & Referencia 1                                                                                                                                                                                                         & Referencia 2                                                                                                                                                                                                                                                             \\ \hline
\begin{tabular}[c]{@{}l@{}}Programación \\ orientada\\  a objetos\end{tabular} & Concepto      & \begin{tabular}[c]{@{}l@{}}Varios. (2003).\\  Programación,\\ Orientada a Objetos.\\  marzo 9, 2018, \\ de openlibra Sitio web:,\\ https://openlibra.com/es/\\ book/programacion\\ -orientada-a-objetos\end{tabular} & \begin{tabular}[c]{@{}l@{}}Velarde de Barraza,\\ Olinda, Murillo de \\ Velásquez Mitzi, \\ Gómez de Meléndez\\  Ludia, Castillo de Krol,\\ Felícita. (2006). \\ INTRODUCCIÓN A \\ LA PROGRAMACIÓN\\  ORIENTADA \\ A OBJETOS. México:,\\ PEARSON EDUCACIÓN.\end{tabular}  \\ \hline
                                                                               & Programa      & \begin{tabular}[c]{@{}l@{}}Joyanes Aguilar Luis\\  (1996). Programación,\\ Orientada a Objetos. \\ 28023 Aravaca (Madrid):\\  McGraw-Hill\end{tabular}                                                               & \begin{tabular}[c]{@{}l@{}}Velarde de Barraza,\\ Olinda, Murillo de \\ Velásquez Mitzi,\\  Gómez de Meléndez \\ Ludia, Castillo de Krol,\\ Felícita. (2006). \\ INTRODUCCIÓN A \\ LA PROGRAMACIÓN\\  ORIENTADA A \\ OBJETOS. México:,\\ PEARSON EDUCACIÓN.\end{tabular}  \\ \hline
                                                                               & Tipo de dato  & \begin{tabular}[c]{@{}l@{}}Joyanes Aguilar Luis\\  (1996). Programación,\\ Orientada a Objetos. \\ 28023 Aravaca (Madrid):\\  McGraw-Hill\end{tabular}                                                               & \begin{tabular}[c]{@{}l@{}}Velarde de Barraza,\\ Olinda, Murillo de \\ Velásquez Mitzi, \\ Gómez de Meléndez \\ Ludia, Castillo de Krol,\\ Felícita. (2006). \\ INTRODUCCIÓN A \\ LA PROGRAMACIÓN \\ ORIENTADA A\\  OBJETOS. México:,\\ PEARSON EDUCACIÓN.\end{tabular}  \\ \hline
                                                                               & Objeto        & \begin{tabular}[c]{@{}l@{}}Varios. (2003).\\  Programación,\\ Orientada a Objetos.\\  marzo 9, 2018, de \\ openlibra Sitio web:,\\ https://openlibra.com/es/\\ book/programacion\\ -orientada-a-objetos\end{tabular} & \begin{tabular}[c]{@{}l@{}}Velarde de Barraza,\\ Olinda, Murillo de \\ Velásquez Mitzi,\\  Gómez de Meléndez\\  Ludia, Castillo de Krol,\\ Felícita. (2006). \\ INTRODUCCIÓN A\\  LA PROGRAMACIÓN \\ ORIENTADA A \\ OBJETOS. México:,\\ PEARSON EDUCACIÓN.\end{tabular}  \\ \hline
                                                                               & Clase         & \begin{tabular}[c]{@{}l@{}}Varios. (2003). \\ Programación,\\ Orientada a Objetos.\\  marzo 9, 2018, de \\ openlibra Sitio web:,\\ https://openlibra.com/es/\\ book/programacion\\ -orientada-a-objetos\end{tabular} & \begin{tabular}[c]{@{}l@{}}Velarde de Barraza,\\ Olinda, Murillo de\\  Velásquez Mitzi,\\  Gómez de Meléndez \\ Ludia, Castillo de Krol,\\ Felícita. (2006). \\ INTRODUCCIÓN A \\ LA PROGRAMACIÓN\\  ORIENTADA A\\  OBJETOS. México:,\\ PEARSON EDUCACIÓN.\end{tabular}  \\ \hline
                                                                               & Herencia      & \begin{tabular}[c]{@{}l@{}}Varios. (2003).\\  Programación,\\ Orientada a Objetos.\\  marzo 9, 2018, de \\ openlibra Sitio web:,\\ https://openlibra.com/es/\\ book/programacion\\ -orientada-a-objetos\end{tabular} & \begin{tabular}[c]{@{}l@{}}Velarde de Barraza,\\ Olinda, Murillo de \\ Velásquez Mitzi, \\ Gómez de Meléndez \\ Ludia, Castillo de Krol,\\ Felícita. (2006).\\  INTRODUCCIÓN A \\ LA PROGRAMACIÓN \\ ORIENTADA A \\ OBJETOS. México:,\\ PEARSON EDUCACIÓN.\end{tabular}  \\ \hline
                                                                               & Polimorfismo  & \begin{tabular}[c]{@{}l@{}}Varios. (2003). \\ Programación,\\ Orientada a Objetos.\\  marzo 9, 2018, de \\ openlibra Sitio web:,\\ https://openlibra.com/es/\\ book/programacion\\ -orientada-a-objetos\end{tabular} & \begin{tabular}[c]{@{}l@{}}Velarde de Barraza,\\ Olinda, Murillo de \\ Velásquez Mitzi, \\ Gómez de Meléndez \\ Ludia, Castillo de Krol,\\ Felícita. (2006). \\ INTRODUCCIÓN A \\ LA PROGRAMACIÓN \\ ORIENTADA A \\ OBJETOS. México:,\\  PEARSON EDUCACIÓN.\end{tabular} \\ \hline
                                                                               & Encapsulación & \begin{tabular}[c]{@{}l@{}}Varios. (2003).\\  Programación,\\ Orientada a Objetos.\\  marzo 9, 2018, de \\ openlibra Sitio web:,\\ https://openlibra.com/es/\\ book/programacion\\ -orientada-a-objetos\end{tabular} & \begin{tabular}[c]{@{}l@{}}Velarde de Barraza,\\ Olinda, Murillo de\\  Velásquez Mitzi, \\ Gómez de Meléndez \\ Ludia, Castillo de Krol,\\ Felícita. (2006). \\ INTRODUCCIÓN A \\ LA PROGRAMACIÓN\\  ORIENTADA A \\ OBJETOS. México:,\\ PEARSON EDUCACIÓN.\end{tabular}  \\ \hline
                                                                               & Abstracción   & \begin{tabular}[c]{@{}l@{}}Varios. (2003).\\  Programación,\\ Orientada a Objetos.\\  marzo 9, 2018, de \\ openlibra Sitio web:,\\ https://openlibra.com/es/\\ book/programacion\\ -orientada-a-objetos\end{tabular} & \begin{tabular}[c]{@{}l@{}}Velarde de Barraza,\\ Olinda, Murillo de \\ Velásquez Mitzi, \\ Gómez de Meléndez \\ Ludia, Castillo de Krol,\\ Felícita. (2006). \\ INTRODUCCIÓN A \\ LA PROGRAMACIÓN \\ ORIENTADA A \\ OBJETOS. México:,\\ PEARSON EDUCACIÓN.\end{tabular}  \\ \hline
                                                                               & Atributos     & \begin{tabular}[c]{@{}l@{}}Joyanes Aguilar Luis\\  (1996). Programación,\\ Orientada a Objetos. \\ 28023 Aravaca (Madrid): \\ McGraw-Hill\end{tabular}                                                               & \begin{tabular}[c]{@{}l@{}}Velarde de Barraza,\\ Olinda, Murillo de \\ Velásquez Mitzi, \\ Gómez de Meléndez \\ Ludia, Castillo de Krol,\\ Felícita. (2006). \\ INTRODUCCIÓN A \\ LA PROGRAMACIÓN\\  ORIENTADA A \\ OBJETOS. México:,\\ PEARSON EDUCACIÓN.\end{tabular}  \\ \hline
                                                                               & Métodos       & \begin{tabular}[c]{@{}l@{}}Joyanes Aguilar Luis\\  (1996). Programación,\\ Orientada a Objetos. \\ 28023 Aravaca (Madrid):\\  McGraw-Hill\end{tabular}                                                               & \begin{tabular}[c]{@{}l@{}}Velarde de Barraza,\\ Olinda, Murillo de \\ Velásquez Mitzi, \\ Gómez de Meléndez \\ Ludia, Castillo de Krol,\\ Felícita. (2006). \\ INTRODUCCIÓN A \\ LA PROGRAMACIÓN \\ ORIENTADA A \\ OBJETOS. México:,\\ PEARSON EDUCACIÓN.\end{tabular}  \\ \hline
                                                                               & Parámetros    & \begin{tabular}[c]{@{}l@{}}Joyanes Aguilar Luis \\ (1996). Programación,\\ Orientada a Objetos. \\ 28023 Aravaca (Madrid): \\ McGraw-Hill\end{tabular}                                                               & \begin{tabular}[c]{@{}l@{}}Velarde de Barraza,\\ Olinda, Murillo de \\ Velásquez Mitzi, \\ Gómez de Meléndez \\ Ludia, Castillo de Krol,\\ Felícita. (2006). \\ INTRODUCCIÓN A\\  LA PROGRAMACIÓN \\ ORIENTADA A \\ OBJETOS. México:,\\ PEARSON EDUCACIÓN.\end{tabular}  \\ \hline
                                                                               & Algoritmo     & \begin{tabular}[c]{@{}l@{}}Joyanes Aguilar Luis\\  (1996). Programación,\\ Orientada a Objetos.\\  28023 Aravaca (Madrid): \\ McGraw-Hill\end{tabular}                                                               & \begin{tabular}[c]{@{}l@{}}Velarde de Barraza,\\ Olinda, Murillo de Velásquez \\ Mitzi, Gómez de Meléndez \\ Ludia, Castillo de Krol,\\ Felícita. (2006). \\ INTRODUCCIÓN A \\ LA PROGRAMACIÓN \\ ORIENTADA A \\ OBJETOS. México:,\\ PEARSON EDUCACIÓN.\end{tabular}     \\ \hline
                                                                               & Variable      & \begin{tabular}[c]{@{}l@{}}Joyanes Aguilar Luis \\ (1996). Programación,\\ Orientada a Objetos. \\ 28023 Aravaca (Madrid): \\ McGraw-Hill\end{tabular}                                                               & \begin{tabular}[c]{@{}l@{}}Velarde de Barraza,\\ Olinda, Murillo de Velásquez \\ Mitzi, Gómez de Meléndez \\ Ludia, Castillo de Krol,\\ Felícita. (2006). \\ INTRODUCCIÓN A \\ LA PROGRAMACIÓN \\ ORIENTADA A \\ OBJETOS. México:,\\ PEARSON EDUCACIÓN.\end{tabular}     \\ \hline
                                                                               & Constante     & \begin{tabular}[c]{@{}l@{}}Joyanes Aguilar Luis \\ (1996). Programación,\\ Orientada a Objetos. \\ 28023 Aravaca (Madrid): \\ McGraw-Hill\end{tabular}                                                               & \begin{tabular}[c]{@{}l@{}}Velarde de Barraza,\\ Olinda, Murillo de \\ Velásquez Mitzi,\\  Gómez de Meléndez \\ Ludia, Castillo de Krol,\\ Felícita. (2006). \\ INTRODUCCIÓN A \\ LA PROGRAMACIÓN \\ ORIENTADA A \\ OBJETOS. México:,\\ PEARSON EDUCACIÓN.\end{tabular}  \\ \hline
\end{longtable}
\newpage
\justify
\textbf{Programación Orientada a Objetos:} La programación orientada a objetos, ha tomado las mejores ideas de la programación estructurada y los ha combinado con varios conceptos nuevos y potentes que incitan a contemplar las tareas de programación desde un nuevo punto de vista. La programación orientada a objetos, permite descomponer más fácilmente un problema en subgrupos de partes relacionadas del problema. Entonces, utilizando el lenguaje se pueden traducir estos subgrupos a unidades auto contenidas llamadas objetos.. \\
\textbf{Programa:} Un programa es una secuencia lógica de instrucciones escritas en un determinado lenguaje de programación que dicta a la computadora las acciones que debe llevar a cabo.
\\
\textbf{Tipo de dato:} es la representación simbólica de un atributo de una entidad; en programación, los datos expresan características de las entidades sobre las que opera un algoritmo. Los datos representan hechos, observaciones, cantidades o sucesos y pueden tomar la forma de números, letras o caracteres especiales.
\\
\textbf{Objeto:} Una estructura de datos y conjunto de procedimientos que operan sobre dicha estructura. Una definición más completa de objeto es: una entidad de programa que consiste en datos y todos aquellos procedimientos que pueden manipular aquellos datos; el acceso a los datos de un objeto es solamente a través de estos procedimientos, únicamente estos procedimientos pueden manipular, referenciar y/o modificar estos datos. 
Para poder describir todos los objetos de un programa, conviene agrupar éstos en clases. 
\\
\textbf{Clase:} Podemos considerar una clase como una colección de objetos que poseen características y operaciones comunes. Una clase contiene toda la información necesaria para crear nuevos objetos. 
\\
\textbf{Encapsulación:} Es una técnica que permite localizar y ocultar los detalles de un objeto. La encapsulación previene que un objeto sea manipulado por operaciones distintas de las definidas. La encapsulación es como una caja negra que esconde los datos y solamente permite acceder a ellos de forma controlada.
\\
\textbf{Abstracción:} En el sentido más general, una abstracción es una representación concisa de una idea o de un objeto complicado. En un sentido más específico, la abstracción localiza y oculta los detalles de un modelo o diseño para generar y manipular objetos.
\\
\textbf{Objetos:} Un objeto es una entidad lógica que contiene datos y un código que manipula estos datos; el enlazado de código y de datos, de esta manera suele denominarse encapsulación. 
Cuando se define un objeto, se está creando implícitamente un nuevo tipo de datos. 
\\
\textbf{Polimorfismo:} Significa que un nombre se puede utilizar para especificar una clase genérica de acciones. 
\\
\textbf{Herencia:} La herencia es un proceso mediante el cual un objeto puede adquirir las propiedades de otro objeto. (Varios , 2003)
\\
\textbf{Atributos:} Describen las características de los objetos: tipo de acceso (privado, protegido, publico) y tipo de dato (entero, real, booleano, etcétera).
\\
\textbf{Métodos:} Describen lo que puede hacer la clase; es decir, el método define las instrucciones necesarias para realizar un proceso o tarea específicos. La definición del método se compone de tipo de acceso, tipo de retorno, nombre del método, parámetros, si los requiere, y el cuerpo del método.
\\
\textbf{Algoritmo:} Se define como una técnica de solución de problemas que consiste en una serie de instrucciones paso por paso y que produce resultados específicos para un problema determinado.
\\
\textbf{Variable:} Es un área de almacenamiento temporal a la que se ha asignado un nombre simbólico y cuyo valor puede ser modificado a lo largo de la ejecución de un programa.
\\
\textbf{Constante:} Es un valor definido que no cambia durante la ejecución de un programa.
\\
\textbf{Parámetros formales:} Son variables que reciben valores desde el punto de llamada que activa al método.
\\
\textbf{Java:} Java es uno de los lenguajes de programación más populares del mundo. Es un lenguaje orientado a objetos, potente, versátil y mutiplataforma (corre en cualquier sistema operativo moderno). Java fue elegido como el lenguaje para el entorno de desarrollo de Android, el sistema operativo móvil líder en smartphones y tablets.
\\
\textbf{Android:} Android es un sistema operativo inicialmente pensado para teléfonos móviles, al igual que iOS, Symbian y Blackberry OS. Lo que lo hace diferente es que está basado en Linux, un núcleo de sistema operativo libre, gratuito y multiplataforma.
\\
\textbf{SQLite:}SQLite es un sistema de gestión de bases de datos relacional compatible con ACID, contenida en una relativamente pequeña (~275 kiB) biblioteca escrita en C.
\\
\textbf{Material Design:} Material Design es un concepto, una filosofía, unas pautas enfocadas al diseño utilizado en Android.
\\
\textbf{Git:} Git es un sistema de control de versiones distribuido cuyo objetivo es el de permitir mantener una gran cantidad de código a una gran cantidad de programadores eficientemente.
\\
\textbf{UML:} UML son las siglas de “Unified Modeling Language” o “Lenguaje Unificado de Modelado”. Se trata de un estándar que se ha adoptado a nivel internacional por numerosos organismos y empresas para crear esquemas, diagramas y documentación relativa a los desarrollos de software.
\\
\textbf{Android Studio:} Es el entorno de desarrollo integrado (IDE) oficial para el desarrollo de aplicaciones para Android y se basa en IntelliJIDEA.


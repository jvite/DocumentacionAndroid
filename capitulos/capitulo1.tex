\chapter{Introducci'on}

\section{Planteamiento del problema}
La programaci'on orientada a objetos es uno paradigma muy importante dentro de la programaci'on por lo tanto resulta muy 'util que los estudiantes que cursan carreras t'ecnicas o universitarias enfocadas al 'ambito de la programaci'on tengan un alto conocimiento de esto, m'as sin embargo a pesar de llevarla como materia, les es dif'icil aprenderla por los amplios conceptos que maneja, por esto nuestro equipo aspira desarrollar una aplicaci'on que sirva para dar apoyo a los alumnos a fin de que refuercen lo que han aprendido en sus clases, as'i como tambi'en puedan aprender cosas nuevas.
Por lo tanto, los alumnos se encuentran ante una cantidad abrumadora de conceptos en un periodo corto de tiempo, lo que dificulta su asimilaci'on y el desarrollo de las habilidades para generar l'ineas de c'odigo como lo explica \cite{spigariol2013ensenando}:

\begin{minipage}{0.9\linewidth}
	 \vspace{5pt}
	 \begin{small}
	 	``Los docentes ve'ian en los estudiantes que el uso del lenguaje representaba una curva de aprendizaje abrupta en los primeros momentos de la materia ya que requieren el manejo de una cantidad 'amplia de conceptos antes de poder realizar algo relativamente sencillo (...). La disociaci'on entre teor'ia y pr'actica que se generaba era ciertamente contraproducente y dificultaba el proceso de aprendizaje."
	 \end{small}
\end{minipage}

\section{Objetivo general}
Desarrollar una aplicaci'on m'ovil que sirva como una herramienta para el aprendizaje del paradigma de programaci'on orientada a objetos con el fin de que cualquier persona pueda acceder a la informaci'on, aprender y/o reforzar su conocimiento sobre este tema, desarrollada a trav'es de Android Studio 

\section{Objetivos espec'ificos} 
\begin{itemize}
\item Desarrollar una aplicaci'on.
\item Proporcionar una herramienta a los estudiantes y/o personas interesadas en la programaci'on.
\end{itemize}
